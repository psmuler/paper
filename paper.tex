\documentclass[./main]{subfiles}
\newboolean{isMain}
\setboolean{isMain}{false}

\begin{document}
\section{Conclusion}

開口部断熱のためのエアロゲル製造時挙動のシミュレーション
(有機無機ハイブリッドキセロゲルの乾燥中挙動とシミュレーション)

はじめに
	研究背景・目的
		社会背景
			開口部には風景を切り取るなど様々な側面があるものの、どのような利用をするにおいても断熱 性能はひろく求められる。特に住宅において、窓部は住宅全体の熱損失の約半分をしめ、窓の断 熱性能を上げることはよりよい住環境を得るうえで必要不可欠である。現状、開口部の断熱につ いて、面材に広く使われているガラスは、一般に断熱性能が低く、高い断熱性を得るためには複 層ガラスが用いられる。しかし、複層ガラスには以下のような課題がある。 
			・空気層の断熱性能はさほど高くなく、一方で断熱性の高い真空や不活性ガスの維持は困難 
			・多層になるにつれて高価、かつ重量が増し取扱いが難しくなる 
			・サッシが特殊になるなど、施工の手間が増える 
			・一枚でも割れると全体を交換する必要があり、非経済
			・可視光透過率や日射熱取得率が低い
			そこで着目したのが、軽量、高断熱、透明な「エアロゲル」という物質である。軽量、高断熱、 高い可視光透過率と優れた物理的特性を示す(表1、エアロゲルの物性値は[1]、三層ガラスの物 性値は[2]より)ため、窓部に用いるのは少ないデメリットで、様々なメリットを得られる。近年の研究により実用化まであと一歩という段階まで来ており、エアロゲルを「夢の材料」ではなく、「新しい標準窓材」にする研究を行いたいと考えた。本研究は卒業論文から続く長期的なエアロゲル実用化および応用に関するものであり、卒業論文で行う大型化手法の研究を深めるとともに、具体的な導入コスト、適用した際の熱負荷軽減量の計算、および極めて軽量であることなどを活かした新しい建築表現の可能性の3点について実験および考察を進める。
			\begin{table}[hbtp]
				\caption{三層ガラスとエアロゲルの物性比較}
				\label{table:data_type}
				\centering
				\begin{tabular}{lS}
				  \hline
				  物性値  & エアロゲル  &  三層ガラス  \\
				  \hline \hline
				  熱貫流率(W/m²・K)  & 0.48 (*1)  & 0.94 \\
				  比重(g/cm³)  & 0.12   & 0.77 (*2) \\
				  可視光透過率(%)  & 95.0(*3)  & 66.2 \\
				  \hline
				\end{tabular}
				[1] tiemfactoryInc.: https://www.tiem.jp/product/material-profile.html, “PRODUCTS-Material Profile”,最 終閲覧日2020/7/20
				[2] 日本板硝子株式会社:三層複層ガラストリプルマルチ-クリア,2018年カタログ
				エアロゲル
				(*1) 三層ガラスと等しい厚み(29mm)について熱伝導率よ り計算
				(*2) ガラスの比重 2.5、ガス封入部を真空として厚さ方 向の平均をとったもの
				(*3) 公表値 近赤外波長800nmにおける値
			\end{table}
		エアロゲルの概要
			エアロゲルは、ゲルに含まれる溶媒を超臨界状態のまま乾燥させ、微細な骨格構造のみを残したものである。 二酸化ケイ素を骨格とするシリカエアロゲルの場合、90パーセント以上が空気であり、最も軽い固体として知られる。
			したがって超臨界乾燥を用いた場合にのみエアロゲルと呼び、それ以外はキセロゲルと呼ぶことも多いが、すでにやが報告しているような常圧下での乾燥でも同様の密度空孔りつを示すゲルが報告されており、一方でキセロゲルの場合にはシリカゲルなどのように骨格を保たずに乾燥させたものも含み多孔性や断熱性に関して全く異なる性質を示す。そのため、本論では高断熱、軽量などの建築開口部に望ましい性質を有するゲルとしてキセロゲルとは区別する意味で、一貫してエアロゲルと記述する。
		常圧乾燥とPMSQについて
			前項で述べたように、有機無機ハイブリッドの三官能基前駆体を用いたPoly Methyle Silses Quioxaneゲル(PMSQゲル)は、常圧であっても適切な乾燥速度下であれば、一度収縮したのちにほぼ収縮前まで復元する「スプリングバック」を起こす。常圧で
			建築で用いられるような大断面の部材はいまだ製造に成功しておらず、
		乾燥収縮について

		乾燥挙動のシミュレーションについて
			乾燥挙動のシミュレーションは有限要素解析法により、
		本研究の目的及び構成
			エアロゲル、特にPMSQエアロゲルは透明断熱材として有用であるが、乾燥時に崩壊してしまうことで歩留まりが低く、大型化が難しい。また同時に、環境的なばらつきに敏感で、形状、湿度などに対して再現性高く実験を行うには非常に多くの工数がかかり、事件的に最適化を行うのは現実的ではない。したがって、理想的な乾燥条件を求めるためのシミュレーション手法が待たれている。
			本論の構成は以下のようになっている。
			1) 従来手法の検討
				エアロゲル/キセロゲルのひび割れ対策、機械強度向上の手法
				乾燥収縮ひび割れを決定するパラメータ
				シミュレーション法
				について既往研究から検討する
			2) パラメータの測定
				ひび割れを決定するパラメータについて、PMSQエアロゲルを対象として測定を行う。
			3) ひび割れ予測
				測定したパラメータをもとに有限要素解析法を用いて形状及び環境をさまざまに変化させたもとでの収縮とひび割れリスクについて予測(シミュレーション)を行う。
			4) 妥当性の検証
				予測した形状から数例を取り上げ、実際の乾燥前後の性状と比較を行う。
	参考文献

既往研究及び文献調査
	エアロゲルのひび割れ対策手法
		エアロゲルのひび割れに対する対策は、乾燥収縮に対するもの以外にもさまざまな手法がとられている。
	エアロゲルの機械的強度の向上手法

	乾燥収縮ひび割れについて
		乾燥収縮ひび割れ
			「乾燥収縮」とはコンクリート中の水が空中に逸散することによって生じる収縮であるが、本論においては、乾燥により収縮が生じるさまざまな多孔質を取り上げて論じる。乾燥による収縮は、物質を構成する大小さまざまな空孔に入っていた水その他の溶媒が揮発する際の表面張力が骨格に及ぶことによって発生し、ひび割れは表層部と内部の水分蒸散の差異によって収縮量の差異が生じることから発生する。
			乾燥収縮ひび割れの研究はさまざまに行われているが、大別してここでは
				1)コンクリート
				2)土
				3)ゲル
			に対するものを取り上げる。

			1)コンクリート
				
			2)土
				\cite{Morris1992CrackingID}によれば土の収縮は土のサクション、圧縮弾性率、ポアソン比、せん断強度、引張強度と、表面エネルギーに依存する。
			3)ゲル
				\cite{SMITH1995191}によるとゲルの乾燥収縮量は密度、\cite{NILSEN1997135}がテトラメトキシシランに対して実験し見ている。
		分散媒の分布
			分散媒の分布は、
		乾燥速度の定式化
		乾燥抵抗
	乾燥収縮ひび割れ予測手法
		建築材料分野では特にコンクリート分野において乾燥収縮のシミュレーションが盛んに行われており、それ以外では食品の
	PMSQの製造
		製造工程
		
		ゲル内部温湿度
		
	
		
	参考文献

乾燥時挙動の測定
	測定法
		試験体組成と形状、調製、養生
		質量測定
		引張試験
			装置の全容
			試験体形状
			静弾性係数の算出

		配筋調査
	測定結果
		
	参考文献

乾燥速度と収縮のシミュレーション
	
	本研究では「AsteaMacs」を用いる。
		
		劣化度とかぶり厚さについて
		雨掛かり量について
		飛来塩分量について
	エアロゲルの乾燥収縮ひび割れ予測
		予測結果の検証の概要
		マルコフ連鎖による遷移確率の算出
		期待劣化度による部材の劣化度分布の予測
		2020年調査結果との比較と考察
	劣化予測手法の精微化
		劣化予測手法の精微化に向けた方針
		屋内環境における飛来塩分量の推定
		鉛直部材と水平部材における遷移確率の違い
		新しい劣化環境区分におけるGE算出式と遷移確率
	劣化予測手法の精度の検討
	まとめ
	参考文献

予測の妥当性検証
	
	端島における維持保全の特殊性1
	劣化における補修補強工法
	補修水準の設定と補修効・補修費用の設定
		補修タイプ1-コンクリート補修
		補修タイプ2-全面的劣化抑制(オーセンティシティへの配慮)
		補修タイプ3-コンクリート補修+全面的劣化抑制
		補修タイプ4-補強
		補修タイプ5-補強+全面的劣化抑制
	維持保全費用と余命延⻑効果の評価
		維持保全費用の定義
		算出結果と考察
	まとめ
	参考文献

建築開口部への適用可能性
	はじめに
		開口部の種類と要求性能
		
	開口部

まとめ
	結論
		劣化予測手法の精微化について
		構造性能の将来予測と余命算出について
		補修方法の検討について
	本研究の他の構造物への応用について
付録


\ifthenelse{\boolean{isMain}}{ %pass 
}{ %else 
    \bibliography{bibtex-demo.bib}
    \bibliographystyle{econ-aea.bst} 
}


\end{document}